To access the eCAD we need to follow few steps. There are also basic requirements which
we need to have to run eCAD. As it works onn both Windows and Ubuntu. So we have
different process for both.
\subsection{For Linux}
\begin{enumerate}
\item Downloading
\begin{itemize}
\item Install Qt libraries using sudo apt-get install qtdeclarative5-dev qt5-default
\item Download zip folder of eCAD or clone it from https://github.com/GreatDevelopers/eCAD
\end{itemize}
\item Installing
\begin{itemize}
\item cd eCAD
\item qmake
\item make
\item ./eCAD
\end{itemize}
\end{enumerate}

\subsection{For Windows}
\begin{enumerate}
\item \textbf{Downloading}: Download zip folder of eCAD from https://github.com/GreatDevelopers/eCAD
\item \textbf{Installing}:  Install Qt’s latest version available with mingw compiler from Qt’s official down-
loads. After installation launch Qt creator load eCAD.pro, from the build menu select
‘Build All”and Run.
\end{enumerate}

